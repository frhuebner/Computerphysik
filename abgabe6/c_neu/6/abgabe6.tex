\documentclass{scrreprt}
\usepackage{etex}
\usepackage[ngerman]{babel}
\usepackage[utf8]{inputenc}
\usepackage[T1]{fontenc}
\usepackage{amsmath, amssymb}
\usepackage{graphicx}

\usepackage{pgfplots}
\pgfplotsset{compat=1.11}
\usepgfplotslibrary{external}
\usepackage{pgfplotstable}

\usepackage{booktabs}
\usepackage{multirow}
\usepackage{longtable}
%\usepackage{ulsy}
%\usepackage{pst-all}
\usepackage{picture}
\usepackage[automark]{scrpage2}
\usepackage{caption}
\pagestyle{scrheadings}
\ihead[]{Friedrich Hübner 2897111}
\ohead[]{Fiona Paulus 2909625} 


\author{Friedrich Hübner 2897111\\
	Fiona Paulus 2909625}
\title{Computerphysik\\Hausarbeit 6}

\begin{document}
	\maketitle
	\newpage
	
\chapter*{Aufgabe 1: Gammalinie}
\section*{allgemeine Hinweise}
Das Programm zu Aufgaben b) und c) wurde unter Linux-Mint mit 'g++ -std=c++11 -g -Wall -Wextra b.cpp -o b.exe' kompiliert.
\section*{a)}
Das Plotten der Datei "profile.dat" ergibt das Spektrum eines Gammastrahlers. Es wird die Impulsrate gegen die Energie aufgetragen.
\begin{center}
	\includegraphics*[scale=0.7]{a.jpeg}
	\captionof*{figure}{Spektrum eines Gammastrahlers}
\end{center}

\section*{b}
Aus dem Spektrum des Gammastrahlers aus a) soll der Kosmische Mikrowellen Hintergrund berechnet werden. Dazu wird eine Spline-Interpolation verwendet, welche mithilfe von festglegten Stützstellen $x_i$ und den dazugehörigen Stützwerten $y_i$ ein Polynom 3.Grades an die Messdaten fittet ("b.cpp").

Dieses Polynom und seine Ableitungen werden für jedes Intervall berechnet.

\[S_i(x)=a_i(x-x_i)^3+b_i(x-x_i)^2+c_i(x-x_i)+d_i\]
\[S'_i(x)=3a_i(x-x_i)^2+2b_i(x-x_i)+c_i\]
\[S''_i(x)=6a_i(x-x_i)+2b_i\]

\[0 \leq i \leq n\]
n: Anzahl der Stützstellen

Um das Gleichungssystem zu lösen werden Bedingungen aufgestellt.
Die Gesamtfunktion muss stetig sein, somit müssen die Polynome der Intervalle an den Stützstellen übereinstimmen.

\[S_i(x_i)=S_{i-1}(x_i)\]
\[S'_i(x_i)=S'_{i-1}(x_i)\]
\[S''_i(x_i)=S''_{i-1}(x_i)\]
Außerdem müssen die Funktionswerte an den Stützstellen mit den zugehörigen Stützwerten übereinstimmen.

\[S_i(x_i)=y_i\]
\[S_{n-1}(x_n)=y_n\]
Und die Krümmung an den Rändern der Gesamtfunktion soll verschwinden.

Die Interpolation wird mit folgenden Stützwerten und dazugehörigen Stützstellen durchgeführt:

\begin{center}
		\begin{tabular}{cccc}
			Stützwert i & Messwert k & Stützstelle $x_i$ & Stützwert $y_i$\\
			\\
			0 & 15 & 50 & 887.992\\
			1 & 415 & 250 & 701.793\\
			2 & 715 & 400 & 623.281\\
			3 & 1015 & 550 & 486.757\\
			4 & 1415 & 750 & 266.316\\
			5 & 1715 & 900 & 209.190\\
		\end{tabular}
\end{center}

Daraus werden die Abstände $h_i=x_{x+1}-x_i$ $0 \leq i \leq 4$ zwischen den Stützpunken berechnet, mit deren Hilfe die Koeffizienten A,B,C,D berechnet werden.

\[A_i = h_{i-1}\]
\[B_i = 2(h{i-1}+h_i)\]
\[C_i = h_i\]
\[D_i = g(\frac{y_{x+1}-y_i}{h_i}-\frac{y_i-y_{i-1}}{h_{i-1}})\]
Daraus werden weitere Koeffizienten berechnet:


\[B'_i = B_i-C_{i-1}\frac{A_i}{B_{i-1}}\]
\[D'_i = D_i-D_{i-1}\frac{A_i}{B_{i-1}}\]
mit $B_1=B'_1$ und $D_1=D'_1$
Außerdem gilt:
\[X_i=y''_i\]
\[X_4 = \frac{D'_4}{B'_4}\]
wodurch sich alle weiteren $X_i$ rekursiv berechnen lassen.

Aus diesen Koeffizienten ergeben sich nun die Koeffizienten der Polynome a,b,c und d.

\[a_i = \frac{y''_{i+1}-y''_i}{6h_i}\]
\[b_i = \frac{y''_i}{2}\]
\[c_i = \frac{y_{i+1}-y_i}{h_i}-\frac{h_i(y''_{i+1}+2y''_i)}{g}\]
\[d_i = y_i\]

Mit diesen Koeffizienten ergeben sich die Polynome mit denen die Funktion des Kosmischen Hintergrunds beschrieben wird.

\begin{center}
	\includegraphics*[scale=0.7]{b.jpeg}
	\captionof*{figure}{Spektrum eines Gammastrahlers und Kosmischer Mikrowellenhintergrund}
\end{center}
\section*{c)}
Nun wird die Funktion des Kosmischen Hintergrunds von der Gammalinie subtrahiert um die reine Strahlung des Gammastrahlers zu erhalten.("b.cpp")

\begin{center}
	\includegraphics*[scale=0.7]{c.jpeg}
	\captionof*{figure}{Spektrum eines Gammastrahlers ohne kosmischen Mikrowellenhintergrund}
\end{center}

\section*{sonstige abgegebene Datein}
\subsection*{plot.plt}
Plotdateien für Aufgabenteile a),b) und c)
\subsection*{cmb.dat}
Ausgabedatei des Programms "b.cpp" für den Aufgabenteil b)
\subsection*{profile-cmb.dat}
Ausgabedatei des Programms "b.cpp" für den Aufgabenteil c)

\end{document}