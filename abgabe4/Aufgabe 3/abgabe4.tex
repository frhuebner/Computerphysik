\documentclass{scrreprt}
\usepackage{etex}
\usepackage[ngerman]{babel}
\usepackage[utf8]{inputenc}
\usepackage[T1]{fontenc}
\usepackage{amsmath, amssymb}
\usepackage{graphicx}

\usepackage{pgfplots}
\pgfplotsset{compat=1.11}
\usepgfplotslibrary{external}
\usepackage{pgfplotstable}

\usepackage{booktabs}
\usepackage{multirow}
\usepackage{longtable}
%\usepackage{ulsy}
%\usepackage{pst-all}
\usepackage{picture}
\usepackage[automark]{scrpage2}
\usepackage{caption}
\pagestyle{scrheadings}
\ihead[]{Friedrich Hübner 2897111}
\ohead[]{Fiona Paulus 2909625} 

\author{Friedrich Hübner 2897111\\
Fiona Paulus 2909625}
\title{Computerphysik\\Hausarbeit 4}

\begin{document}
\maketitle
\newpage

\chapter*{Der Differentialgleichungssolver}
Für den DGL-Solver wird das Fehlbergverfahren mit automatischer Schrittweitensteuerung aus der Vorlesung verwendet. Die Berechnung aller Werte erfolgt genau nach diesem Verfahren.

\section*{Programm (abgabe4_runge_kutta.cpp)}


\chapter*{Aufgabe 1}


\section*{Sonstige abgegebene Dateien}
\subsection*{plot\_bessel.plt}
Die Plot-Datei für die Besselfunktionen
\subsection*{plot\_slit.plt}
Die Plot-Datei für die Kreisblende
\subsection*{slit.txt}
Enthält die Plotdaten für die b)
\end{document}